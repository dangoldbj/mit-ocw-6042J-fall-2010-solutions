\documentclass[titlepage,12pt]{book}
\usepackage[utf8]{inputenc}

\usepackage[margin=1in]{geometry}
\usepackage[colorlinks,linkcolor=blue,urlcolor=blue]{hyperref}
\usepackage{ntheorem}
\usepackage{enumitem}
\usepackage{amsmath}
\usepackage{amssymb}
\usepackage{tikz}


% configure ntheorem
\theoremstyle{break}
\theorembodyfont{\upshape} % no italics in body font
\newtheorem{exercise}{Exercise}
\newtheorem{solution}{Solution}
\newtheorem{lemm}{Lemma}
\newtheorem{corol}{Colorally}
\newtheorem*{theorem}{Theorem}
\newtheorem*{proof}{Proof}

% shortcuts
\newcommand{\enum}[1]{\begin{enumerate}[label=(\alph*)] #1 \end{enumerate}}
\newcommand{\enumr}[1]{\begin{enumerate}[label=(\roman*)] #1 \end{enumerate}}

\newcommand{\TODO}{\textbf{TODO }}
\newcommand{\heart}{\ensuremath\heartsuit}
\newcommand{\blank}{\makebox[1cm]{\hrulefill}}
\newcommand{\blankk}{\blank\space}

% math shortcuts
\newcommand*\closure[1]{\overline{#1}}
\newcommand{\abs}[1]{\left|#1\right|}



\begin{document}

\mainmatter

\begin{solution}
    Sequence $a_{1},a_{2},a_{3},a_{4},a_{5}$
    \enum {
        \item
            \begin{proof} 
                By Contradiction.
                Suppose that there is no 3-chain in our sequence and $a_{1} \leq a_{3}$.
                If $a_{4} \geq a_{3}$ then we have 3-chain with $a_{1}, a_{3}, a_{4}$.
                This implies $a_{4} < a_{3}$.
                
                If $a_{4} < a_{3}$ then $a_{1} < a_{2}$ and $a_{1} \leq a_{3}$ implies $a_{3} < a_{2}$.
                So, $a_{4} < a_{3} < a_{2}$ is a 3-chain.
                This implies $a_{4} \geq a_{3}$.
                Contradiction.

                Thus $a_{1} > a_{3}$.
            \end{proof}

        \item 
            \begin{proof}
                If there is no 3-chain then $a_{1} > a_{3}$.
                So, $a_{3} < a_{2}$. Now, $a_{4} >  a_{3}$ and $a_{4} < a_{2}$ for no 3-chain to exist.
                Thus, $a_{3} < a_{4} < a_{2}$.
            \end{proof}

        \item
            \begin{proof}
                We have $a_{1} < a_{2}$ and $a_{3} < a_{4} < a_{2}$.

                $a_{5} \geq a_{2}$ will result in 3-chain $a_{1}, a_{2}, a_{5}$
                $a_{5} \geq a_{4}$ will result in 3-chain $a_{3}, a_{4}, a_{5}$
                $a_{5} \geq a_{1}$ will result in 3-chain $a_{3}, a_{4}, a_{5}$ or $a_{2}, a_{4}, a_{5}$.
                $a_{5} \geq a_{3}$ will result in 3-chain $a_{2}, a_{4}, a_{5}$.
                $a_{5} \leq a_{3}$ will result in 3-chain $a_{2}, a_{3}, a_{5}$.

                Thus any value of $a_{5}$ produces a 3-chain.
            \end{proof}

        \item
            \begin{proof}
                By Contradiction.
                Suppose $\exists a_{1},a_{2},a_{3},a_{4},a_{5}$ such that no 3-chain exists.

                If $a_{1} > a_{2}$ then,
                $a_{2} < a_{3} < a_{1}$ then any $a_{4}$ will create 3-chain.

                $a_{2} < a_{1} < a_{4} < a_{3}$ any $a_{5}$ will create 3-chain.


                If $a_{1} < a_{2}$ then, $a_{1} > a_{3}$ and $a_{3} < a_{4} < a_{2}$.
                But now, any $a_{5}$ will create a 3-chain.

                Contradiction.
            \end{proof}
    }
\end{solution}

\begin{solution}
    By Induction.
    
    
    Induction Hypothesis: $P(n)$ implies for all non negative integer $n$, 
    
    $\sum_{i=0}^{n} i^{3} = ((n(n+1)) / 2)^{2}$

    
    Base case: $n = 0$. $P(0)$ is $\sum_{i=0}^{0} i^{3} = 0 = ((0(0+1)) / 2)^{2}$.
    Thus, $P(0)$ is true.

    For induction, assume $P(n)$ is true.
    Now, $P(n+1)$ is $\sum_{i=0}^{n+1} i^{3} = (((n + 1)(n+2)) / 2)^{2}$.

    Then, $\sum_{i=0}^{n+1} i^{3} = \sum_{i=0}^{n} i^{3} + (n + 1)^{3}$.

    $((n(n+1))/2)^{2} + (n + 1)^{3} = (n + 1)^{2}(n^{2}/4 + (n + 1)) = (n + 1)^{2}((n^{2}+ 4n + 4)/4) = (n + 1)^{2}((n+ 2)^{2}/4) = (((n + 1) (n + 2)) / 2)^{2}$.

    Therefore $P(n) \implies P(n+1)$.

    By the axiom of induction $P(n)$ is true.

\end{solution}


\begin{solution}
    By induction.
    
    Suppose that the num edges can reach beyond the grid.
    Then for any $m < n$, num edges is atmost $4m$ which is less than $4n$.
    And for full grid num edges is exactly $4n$.

    Let $x$ denote the num edges.

    States:
    If a new square is infected 
        it will cancel at least two edges and add at most two edges.
    After any legal move the state of the grid will change in following ways:

    \enumr {
        \item $x$ will remain unchanged.
        \item $x$ will decrease by 1.
        \item $x$ will decrease by 2.
        \item $x$ will decrease by 4.
    }

    Induction Hypothesis: $P(m)$ implies after $m$ time-steps, 
        $x < 4n$ for $n \times n$ grid.

    Base Case: $P(0)$ is true. Because, $m < n$ and $x \leq 4m < 4n$.

    Assume it is true for all $m$ for purposes of induction.

    After $m + 1$ steps the state changes. But the new state  $x'$
    will be $x' \leq x$. Since $x < 4n$ (from induction hypothesis).
    $x' < 4n$.

    It follows that $P(m) \implies P(m + 1)$.
    Thus $P(m)$ is true.
\end{solution}

\begin{solution}
    The inductive hypothesis only covers $a^{k}$ not $a^{-1}$.
    Assuming $a^{-1} = 1$ requires base case to consider $k = 1$.
    But the base case only considers $k = 0$.
    Simply assuming $a^{-1} = a^{1} = 1$ assumes $P(k + 1)$ is true implicitly.
    But the proposition $P(k + 1)$ is the thing we are trying to prove.
\end{solution}

\begin{solution}
    By Induction:
    $P(n)$ be $G_{n} = 3^{n} - 2^{n}$.

    Special Case: For $n = 0$. $G_{0} = 0$ which is true.

    Base Case: For $n = 1$. $G_{1} = 1$ which is true.

    Inductive Step: Assume $\forall n \geq 1, P(n)$ is true.

    $P(n + 1)$ is $G_{n + 1} = 3^{n+1} - 2^{n + 1}$.
    So, $G_{n + 1} = 5 G_{n} - 6 G_{n - 1} 
    = 5 (3^{n} - 2^{n}) - 6 (3^{n - 1} - 2^{n - 1})
    = 5 \cdot 3^{n} - 5 \cdot 2^{n} - 2 \cdot 3^{n} + 3 \cdot 2^{n}
    = 3 \cdot 3^{n} - 2 \cdot 2^{n}
    = 3^{n + 1} - 2^{n + 1}
    $

    which implies that $P(k + 1)$ holds.
    If follows by induction that $P(k)$ holds for all $k \in \mathbb{N}$.


\end{solution}

\end{document}

%%% Local Variables:
%%% mode: latex
%%% TeX-master: t
%%% End:
