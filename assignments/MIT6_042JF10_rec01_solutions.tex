\documentclass[titlepage,12pt]{book}
\usepackage[utf8]{inputenc}

\usepackage[margin=1in]{geometry}
\usepackage[colorlinks,linkcolor=blue,urlcolor=blue]{hyperref}
\usepackage{ntheorem}
\usepackage{enumitem}
\usepackage{amsmath}
\usepackage{amssymb}
\usepackage{tikz}


% configure ntheorem
\theoremstyle{break}
\theorembodyfont{\upshape} % no italics in body font
\newtheorem{exercise}{Exercise}
\newtheorem{solution}{Solution}
\newtheorem{lemm}{Lemma}
\newtheorem{corol}{Colorally}
\newtheorem*{theorem}{Theorem}
\newtheorem*{proof}{Proof}

% shortcuts
\newcommand{\enum}[1]{\begin{enumerate}[label=(\alph*)] #1 \end{enumerate}}
\newcommand{\enumr}[1]{\begin{enumerate}[label=(\roman*)] #1 \end{enumerate}}

\newcommand{\TODO}{\textbf{TODO }}
\newcommand{\heart}{\ensuremath\heartsuit}
\newcommand{\blank}{\makebox[1cm]{\hrulefill}}
\newcommand{\blankk}{\blank\space}

% math shortcuts
\newcommand*\closure[1]{\overline{#1}}
\newcommand{\abs}[1]{\left|#1\right|}



\begin{document}

\mainmatter

\begin{solution}
\enum {
    \item $\exists x \in X$ such that $S(x) \land A(x)$ is true.
    \item $\forall x \in X, T(x) \land S(x) \implies A(x)$
    \item $\nexists x \in X$ such that $(T(x) \land S(x)) \land \neg A(x)$
    \item $S = \{ \forall x \in X : T(x) \land \neg S(x) \}$. Then, $|S| \geq 3$
}
\end{solution}


\begin{solution}
    TODO: It's easy enough. So, I'll skip it for later.
\end{solution}

\begin{solution}
    \enum {
        \item \enumr {
            \item $\neg (A$ nand $B)$
            \item $(A$ nand $A)$ nand $(B$ nand $B)$
            \item $(\neg A$ nand $\neg A)$ nand $(B$ nand $B)$
        }

        \item $\neg A = A$ nand $A$.

        \item true is $A$ nand $(A$ nand $A)$. 
            Similarly, false is $(A$ nand $(A$ nand $A))$ nand $(A$ nand $(A$ nand $A))$
    }
\end{solution}

\begin{solution}
    Split the 12 coins in two groups of 6.
    Weigh the two groups in the balance scale.
    Since one of the coin is heavier the 
    balance scale will tilt towards the heavy side.
    Now split the 6 coins in the heavy side into two groups of 3.
    Weigh the two groups in the balance scale.
    Since one of the coin is heavier the 
    balance scale will tilt towards the heavy side.
    Now split the 3 coins in the heavy side into three groups of 1.
    Put any of the two coins on the balance scale. Now only two cases can arise:
    \enumr {
        \item Balance scale is balanced. This implies the coin that was not kept in the balance scale is the heavier coin.
        \item Balance scale tilts towards one of the sides. The heavier side has the heavy coin.
    }
\end{solution}

\begin{solution}
    By Contrapositive: If $r^{1/5}$ is not irrational then $r$ is not irrational.
    \begin{proof}
        Let, $r^{1/5} = (a/b)$. Then, $r = (a/b)^{5} = a^{5}/b^{5}$.
        $a, b \in \mathbb{Z} \implies a^{5}, b^{5} \in \mathbb{Z}$.
        Thus $r \in \mathbb{Q}$.
    \end{proof}
\end{solution}

\begin{solution}
    $w^{2} + x^{2} + y^{2} = z^{2}$.
    % Let, $O(x)$ and $E(x)$ be predicates for $x$ is odd and $x$ is even respectively.
    % \begin{lemm}[Even + Odd is Odd. Even + Even is Even. Odd + Odd is Even]
    %     Even plus odd is odd.
    %     $e = 2i, o = 2j + 1$ so, $e + o = 2i + 2j + 1$ which is odd.

    %     Odd plus Odd is even.
    %     $e = 2i + 1, o = 2j + 1$ so, $e + o = 2i + 2j + 2$ which is even.

    %     Even + Even is even.
    %     $e = 2i, o = 2j$ so, $e + o = 2i + 2j$ which is even.
    % \end{lemm}
    % \begin{lemm}[$\forall n \in \mathbb{N} - \{0\}, x \in \mathbb{Z} : O(x) \implies x^{n} \in \mathbb{Z}: O(x^{n})$.]
    %     By induction. Let, $x = 2j + 1$.
    %     For $n = 0$, $x^{1} = x$ which is odd.
    %     Assume inductively that it is true for all $n$.
        
    %     Now, $x^{n + 1} = x * x^{n} = (2j + 1) * o$ for some odd number $o$.
    %     $2jo + o$ is odd since $2jo$ is even and $o$ is odd. Even plus odd is odd.
    % \end{lemm}
    % \begin{lemm}[$\forall n \in \mathbb{N} - \{0\}, x \in \mathbb{Z} : E(x) \implies x^{n} \in \mathbb{Z}: E(x^{n})$.]
    %     By induction. Let, $x = 2j$.
    %     For $n = 0$, $x^{1} = x$ which is even.
    %     Assume inductively that it is true for all $n$.
        
    %     Now, $x^{n + 1} = x * x^{n} = (2j) * e$ for some even number $e$.
    %     $2je$ is even.
    % \end{lemm}
    \begin{proof}
        For the forward implication:
        Consider $w, x, y$ are even. Then,
        $w=2i_{1}, x=2i_{2}, y=2i_{3}$.
        So, $w^{2} + x^{2} + y^{2} = 2^{2}(i_{1}^{2}+i_{2}^{2}+i_{3}^{2}) = z^{2}$.
        Since, $z \in \mathbb{Z^{+}}$, any integer solution to $2^{2}(i_{1}^{2}+i_{2}^{2}+i_{3}^{2}) = z^{2}$ will be divisible by $2$.
        Thus $z$ is even.
        
        
        For the backward implication:
        If $z = 2j$. Then $2^{2}j^{2} = z^{2} = w^{2} + x^{2} + y^{2}$ gives
        $j^{2} = (w/2)^{2} + (x/2)^{2} + (y/2)^{2}$.

        Since $j \in \mathbb{Z^{+}} \implies j^{2} \in \mathbb{Z^{+}}$.
        This implies that $w, x, y$ has to be divisible by $2$ so they are even.

    \end{proof}
\end{solution}

\end{document}

%%% Local Variables:
%%% mode: latex
%%% TeX-master: t
%%% End:
