\documentclass[titlepage,12pt]{book}
\usepackage[utf8]{inputenc}

\usepackage[margin=1in]{geometry}
\usepackage[colorlinks,linkcolor=blue,urlcolor=blue]{hyperref}
\usepackage{ntheorem}
\usepackage{enumitem}
\usepackage{amsmath}
\usepackage{amssymb}
\usepackage{tikz}


% configure ntheorem
\theoremstyle{break}
\theorembodyfont{\upshape} % no italics in body font
\newtheorem{exercise}{Exercise}
\newtheorem{solution}{Solution}
\newtheorem{lemm}{Lemma}
\newtheorem{corol}{Colorally}
\newtheorem*{theorem}{Theorem}
\newtheorem*{proof}{Proof}

% shortcuts
\newcommand{\enum}[1]{\begin{enumerate}[label=(\alph*)] #1 \end{enumerate}}
\newcommand{\enumr}[1]{\begin{enumerate}[label=(\roman*)] #1 \end{enumerate}}

\newcommand{\TODO}{\textbf{TODO }}
\newcommand{\heart}{\ensuremath\heartsuit}
\newcommand{\blank}{\makebox[1cm]{\hrulefill}}
\newcommand{\blankk}{\blank\space}

% math shortcuts
\newcommand*\closure[1]{\overline{#1}}
\newcommand{\abs}[1]{\left|#1\right|}



\begin{document}

\mainmatter

\begin{solution}
\enum {
    \item $\exists x \in X$ such that $S(x) \land A(x)$ is true.
    \item $\forall x \in X, T(x) \land S(x) \implies A(x)$
    \item $\nexists x \in X$ such that $(T(x) \land S(x)) \land \neg A(x)$
    \item $S = \{ \forall x \in X : T(x) \land \neg S(x) \}$. Then, $|S| \geq 3$
}
\end{solution}


\begin{solution}
    TODO: It's easy enough. So, I'll skip it for later.
\end{solution}

\begin{solution}
    \enum {
        \item \enumr {
            \item $\neg (A$ nand $B)$
            \item $(A$ nand $A)$ nand $(B$ nand $B)$
            \item $(\neg A$ nand $\neg A)$ nand $(B$ nand $B)$
        }

        \item $\neg A = A$ nand $A$.

        \item true is $A$ nand $(A$ nand $A)$. 
            Similarly, false is $(A$ nand $(A$ nand $A))$ nand $(A$ nand $(A$ nand $A))$
    }
\end{solution}

\begin{solution}
    Split the 12 coins in two groups of 6.
    Weigh the two groups in the balance scale.
    Since one of the coin is heavier the 
    balance scale will tilt towards the heavy side.
    Now split the 6 coins in the heavy side into two groups of 3.
    Weigh the two groups in the balance scale.
    Since one of the coin is heavier the 
    balance scale will tilt towards the heavy side.
    Now split the 3 coins in the heavy side into three groups of 1.
    Put any of the two coins on the balance scale. Now only two cases can arise:
    \enumr {
        \item Balance scale is balanced. This implies the coin that was not kept in the balance scale is the heavier coin.
        \item Balance scale tilts towards one of the sides. The heavier side has the heavy coin.
    }
\end{solution}

\begin{solution}
    
\end{solution}

\end{document}

%%% Local Variables:
%%% mode: latex
%%% TeX-master: t
%%% End:
